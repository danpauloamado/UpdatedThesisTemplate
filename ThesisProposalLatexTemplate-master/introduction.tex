\section{Background of the Study}

The Philippines is the world’s eighth-largest rice producer. Its arable land totals 5.4 million hectares. Rice area harvested has expanded from nearly 3.8 million hectares in 1995 to about 4.4 million hectares in 2010. However, the country’s rice area harvested is still very small compared with that of the other major rice-producing countries in Asia. Climate change, growing population, declining land area, high cost of inputs, and poor drainage and inadequate irrigation facilities are the major constraints to rice production in the Philippines. Some of these constraints are interrelated. Unabated conversion of some agricultural land to residential, commercial, and industrial land reduces the area devoted to rice production, which leads to a shortage in domestic supply (ricepedia.org). The Philippines is one of the largest producers of rice in the world, despite of having an inadequate rice area caused by several factors which led to inadequacy of domestic supply.   

The Philippines imports about 10\% of its annual consumption requirements. In 2010 and 2011, the country was the biggest rice importer. Its rice imports amounted to 2.38 million t in 2010, mostly coming from Vietnam and Thailand. (ricepedia.org). Despite of being one of the largest rice producers in the world, the Philippines still imports rice from their neighboring countries to make up for the shortage in its domestic supply.

For the Philippines to become self-sufficient in rice, it has to adopt existing technologies such as improved varieties and know-how to have yield increase by 1–3 t/ha. Better quality seed combined with good management, including new postharvest technologies, is the best way to improve rice yields and the quality of production (ricepedia.org). The utilization of new technology could help increase the production of rice in the country, increase our domestic supply, decrease the need to import rice, reduce the consumer cost, and increase the profit gain of farmers.


\section{Prior Studies}

A resource entitled "A Robot System for Paddy Field Farming in Japan" is set to utilize a robot-operated farming technology guided from tillage to harvest in large-scale agriculture. In such application, it is seen that in the cultivation of rice, wheat and soybean (in Japan, as per the researchers' host country), there has been three types of robot in development. First, a robot tractor, followed by a rice transplanter, finally, combines harvester robots. Real-time Kinematic Global Positioning System (RTK-GPS) and Inertia Measurement Unit (IMU), or Global Positioning System (GPS) compass are utilized for navigation system. These robots have a Controller Area Network (CAN) bus that all sensors and computers can be connected and interfaced in common among other robots such as tractors, rice transplatners and combine harvesters. Hence, these could be officiated in autonomous operation in paddy fields as well as discussing in this paper the ability of moving across fields for effective operations and safe guidelines for robot systems.

Another is a resource entitled “A Global Positioning System guided automated rice transplanter" that speaks about a new Global Positioning System (GPS) guided rice transplanter. This study is very coherent to the aforementioned research as this resource speaks more about the utilization of the GPS technology they used in implementing the three robots as tractor, rice transplanter and combine harvester. With these, such robot systems were GPS-guided with their respective position data and inertia measurement unit direction data. This new one (inherent to this resource) is guided with GPS position data with tilt correction during straight driving and guided with the data gathered from the IMU during each robot's turning at the head land. An antenna prescribed to the GPS is set to 1.5 meters (as height) and 0.4 meters as its offset at the vehicle's front axle. The actuator control command and data communication protocols adhere through the controller area network (CAN) bus. Hence, steering and transmission systems are controlled through electrical actuators with respect to the location in a given field.

Lastly, a resource entitled “Robot Farming System Using Multiple Tractors in Japan” with the objective to develop a robot farming system using multiple robots. It discusses the application of multiple robots in Japan agriculture for rice, wheat, and soybean. The system that is discussed in this paper includes a rice planting robot, a seeding robot, a robot tractor, a combine robot harvester, and several tools attached on the robot tractor. The main objective of this paper is to help the farmers gain more profit thru farming. The paper focused on robot management system, low-cost system, robot farming safety, and real-time monitoring/documentation.




\section{Problem Statement}
The Philippines is rich in fertile lands suitable for agricultural development. However, due to the absence of advanced tools for farming, rice shortage is becoming a problem. Filipinos are importing rice from other countries such as Thailand and Vietnam in spite of the capability of the Philippine land to cultivate rice.
 
Philippine farmers are not equipped with tools that could compete with the advanced instruments used by foreign farmers. Most of the Philippine farmers rely on manual labor. Difficult tasks such as sowing the field are done by the farmers yet their salary is still below the minimum wage. The land may be rich and fertile for agriculture but the agricultural sector, specifically the local farmers, are considered one of the poorest sector in the country. In turn, the rice fields are neglected. According to National Geographic, “Some 25 to 30 percent of the terraces are abandoned and beginning to deteriorate, along with irrigation systems”. Investors and laborers are avoiding the agricultural industry due to the absence of advanced systems used in planting rice.
 
\section{Objectives}
\subsection{General Objective(s)}
To design and develop a system that would automate plantation of rice in paddy fields in the Philippines;
 
\subsection{Specific Objectives}
 
\begin{enumerate}
\item To implement computer vision, specifically edge detection, in tracing the path sections of the paddy field;
 
\item To utilize the flood fill algorithm in designing the optimal route for the mobile robot as it plant the rice;
 
\item To design an Arduino system in implementing computer vision as interface in robotic application;
 
\item To design and develop a mobile robot designed to withstand paddy field environmental factors (e.g. soil, mud, etc.);
 
\end{enumerate}
 
\section{Significance of the Study}
 
Computer Engineering is the marriage of electronics and programming. Implementing a programming-based instruction on an electronic hardware is a fundamental action in the progression of this course. With the use of programming, hardware systems are automated with a more defined set of instructions. With this, the study of a Robot System for the Paddy Field in the Philippines would be an unwavering focus related to the field. The implementation of this robot system would reinforce automation with the aid of computer vision. Moreover, the electronic and programming skills of the students would be strengthened with this research. External elements such as the edge of the paddy field increase the complexity of this longstanding research. Robot systems are no longer fairly new. However, introducing computer vision that would direct a robot system that could withstand environmental factors, specifically in paddy fields, would establish an innovation for the field of Computer Engineering and for the country Philippines as well.
 
In social context, the employment of this robot system for paddy field planting would allow a decrease in production time of rice as it automates the planting of the crop. Additionally, it would lessen the manual labor provided by the local farmers. Instead of manually planting rice, local farmers would save time and effort as the robot system for paddy field planting would be utilized. The workload for the farmers would be decreased as the production is increased. It is anticipated that the use of this system would increase the productivity of agricultural sector in the country. It would aide local farmers in ensuring an increase in rice yield as plantation is automated. It will not only benefit the agricultural area but also the economic status of the Philippines.
 
By engaging software-heavy technique such as computer vision into an electronic device, this research would be principal in establishing further the discipline of Computer Engineering. Considering programming as the automation mechanism of systems would yield a better and more accurate result as the set of instructions is broadened. This research is also essential in developing the programming and hardware skills of the students. Simultaneously, this research is significant due to the demand of increasing the competency of the agricultural sector of the Philippines.





\section{Assumptions, Scope and Delimitations}

Bulletize your scope in one group, and then bulletize the delimitations in another.  Bulletize your assumptions as well.


\section{Description and Methodology}

\blindtext


\ifFinished
\else

\section{Estimated Work Schedule and Budget}

Gantt chart or similar is to be part of this section.

\blindtext

\section{Publication Plan}
\blindtext

\fi


\section{Overview}

Provide here a brief summary and what the reader should expect from each succeeding chapter.  Show how each chapter are connected with each other.

